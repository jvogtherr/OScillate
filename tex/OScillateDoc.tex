\documentclass[12pt,a4paper]{scrartcl}
\usepackage[utf8]{inputenc}
\usepackage[T1]{fontenc}
\usepackage{lmodern}
\usepackage[ngerman]{babel}
\usepackage{amsmath}
\usepackage[automark, headsepline, footsepline]{scrpage2}
\usepackage{graphicx}
\usepackage{url}

\widowpenalty10000
\ihead[]{}
\chead[]{\headmark}
\ohead[]{}
\ifoot[]{}
\cfoot[\pagemark]{\raisebox{-.2cm}{\thepage}} 
\ofoot[]{}
\pagestyle{scrheadings}
\setkomafont{pageheadfoot}{\normalfont\small\sf}

\author{Jan Philipp Vogtherr \& Timmy Schüller\vspace{0.5cm}}
\title{\includegraphics[scale=0.8]{unilogo.pdf}\vspace*{1cm}
\mbox{Projekt OScillate:} \mbox{Ein Modell der Osnabrücker} Buslinien 11 und 21\vspace{0.3cm}}
\date{Universität Osnabrück \\
Im Rahmen der Veranstaltung \glqq Regelbasierte Modellierung\grqq \\
\vspace*{0.4cm}
Wintersemester 2013/2014 \\
\today}

\begin{document}

\maketitle
\thispagestyle{empty}
\newpage
\tableofcontents
\newpage

\section{Motivation}
Als Student an der Universität Osnabrück bzw. der Hochschule steht man täglich vor einer schwierigen Entscheidung, wenn man den Kampus am Westerberg erreichen möchte. Der Weg dorthin beinhaltet nämlich für viele eine Busfahrt vom zentralen Verkehrsknotenpunkt, dem Neumarkt, zum Zielort, wobei es dort zwei verschiedene Linien gibt, die jeweils eigene Vor- und Nachteile mit sich bringen. Die Buslinie 21 fährt einen kleinen Umweg, sodass sie langsamer ist, als die konkurrierende Buslinie 11. Außerdem fährt die 11 mit einem 10-Minuten Takt häufiger als die 21. Allerdings hält die 11 nicht direkt an der Haltestelle vor der Uni, sondern zieht immer einen Fußweg von maximal fünf Minuten mit sich. Die zu Beginn angesprochene Entscheidung ist also nun: \glqq Fahre ich heute mit der 11, oder der 21?\grqq

Das im folgende vorgestellte Projekt OScillate beschäftigt sich  mit ebendieser Problematik. Im Rahmen der Veranstaltung \glqq Regelbasierte Modellierung\grqq  wurde ein agentenbasiertes Modell erstellt, das die Entscheidungsdynamik %bessere Formulierung?
zwischen den konkurrierenden Buslinien 11 und 21 über einen annähernd normalen Tagesablauf, dh. über die Zeit, nachstellen soll.

In Abschnitt~\ref{design} werden zunächst grundlegende Designentscheidungen erläutert. Darauf basierend geht es in Abschnitt~\ref{impl} um die Umsetzung der Ideen, direkt gefolgt von einer Auswertung der daraus resultierenden Ergebnisse in Abschnitt~\ref{erg}. Abschließend diskutiert Abschnitt~\ref{fazit} den Verlauf des Projektes, sowie die erhaltenen Ergebnisse.

\section{Design}\label{design}
Das Ziel des Modelles sind von explorativer Natur und lässt sich in folgender Leitfrage zusammenfassen: \glqq Welche Faktoren haben einen Einfluss auf die Wahl der Buslinie?\grqq 

Da das Modell, um die Busfahrzeiten und Tagesabläufe eines Studenten darstellen zu können,  eine sehr hohe zeitliche Auflösung mit sich bringt, kann der Abstraktionsgrad als realistisch eingestuft werden. Die komplexe Entscheidung der homogenen Agenten, der Studenten, trägt ebenfalls dazu bei und führt zu einer hohen individuellen Komplexität. Funktionale Heterogenität gibt es in direktem Sinne nicht, weil es außer den Studenten keine weiteren Akteure gibt. Eine Interaktion zwischen Agenten ist ebenfalls minimal, denn einen direkten Austausch von Erfahrungen oder Ähnlichem zwischen den Studenten gibt es nicht.

\section{Implementation}\label{impl}

\section{Auswertung}\label{erg}

\section{Fazit}\label{fazit}

\end{document}